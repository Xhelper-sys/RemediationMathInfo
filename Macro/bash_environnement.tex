\documentclass{article}
\usepackage[utf8]{inputenc}
\usepackage[T1]{fontenc}
\usepackage{xcolor}
\usepackage{listings}

\begin{document}

\section{Exemple de code Bash stylisé}

% J'ai compacté les options pour éviter les espaces invisibles problématiques
\begin{lstlisting}[language=bash, basicstyle=\small\ttfamily\bfseries, columns=fixed, keywordstyle=\color{blue}\bfseries, stringstyle=\color{orange}, commentstyle=\color{gray}\itshape, frame=single, breaklines=true, showstringspaces=false, tabsize=2, gobble=4]
    # 1. Creation de l'arborescence
    mkdir repoC
    cd repoC
    
    # 2. Creation des fichiers
    touch correction.tex
    echo "Ceci est un test" > fichier.txt

    # 3. Versionnage avec Git
    git add .
    git commit -m "Test"
\end{lstlisting}

\end{document}