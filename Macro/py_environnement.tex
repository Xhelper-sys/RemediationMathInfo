\documentclass{article}
\usepackage[utf8]{inputenc}
\usepackage[T1]{fontenc}
\usepackage{xcolor}    % Indispensable pour les couleurs
\usepackage{listings}  % Indispensable pour le code

\begin{document}

\section{Exemple Python}

Voici un exemple de classe en Python avec coloration syntaxique :

% J'ai compacté les options pour éviter le bug des lignes vides
\begin{lstlisting}[language=Python, basicstyle=\small\ttfamily, keywordstyle=\color{blue}\bfseries, stringstyle=\color{orange}, commentstyle=\color{green!50!black}\itshape, frame=single, numbers=left, numberstyle=\tiny\color{gray}, showstringspaces=false, tabsize=4, breaklines=true]
import math

class Calculateur:
    """ Ceci est une classe de demonstration """
    
    def __init__(self, valeur):
        self.valeur = valeur
        
    def racine_carree(self):
        # On utilise la bibliotheque math
        if self.valeur < 0:
            return None
        return math.sqrt(self.valeur)

# Test du programme
if __name__ == "__main__":
    mon_calc = Calculateur(25)
    print(f"La racine est : {mon_calc.racine_carree()}")
\end{lstlisting}

\end{document}