\documentclass{article}

\usepackage[utf8]{inputenc}
\usepackage[T1]{fontenc}
\usepackage[french]{babel}
\usepackage{mathtools}
\usepackage{amssymb}
\usepackage{blindtext}
\usepackage{hyperref}

\title{Des maths, des maths!}
\author{Mustafa Kaya}
\date{\today}

\begin{document}
\begin{titlepage}
    \maketitle
\end{titlepage}

%Question 2: Le rendus n est pas beau
3-2+4=5

%Question 3: Le rendus est beaucoup plus classe
$3-2+4=5$

%Qestion 4
% En mode normale le resultat provoque une erreur
% tandis que en mode math c'est mis a la puissance

$2^{10}=1024$

%Qestion 5
$a_{1} \times a_{2} \times \ldots\times a_{n}$

%qestion 6:
$a_{1} \times a_{2} \times \cdots \times a_{n}$

%Question 7
$ 3 < 4 , 5 > 4$
$\leq$ % Plus petit ou egale
$\geq$ % Plus grand ou egale

%Question 8
$\exists{x}, \forall{n}, \neq, \iff, \sim, \hookrightarrow$
$\cos{\theta}, \sin{\theta}, \tan{\theta} \lim$
$\frac{1}{7}$
$\infty$

%Qestion 9
%mathtools paquets
$\implies$

%amssymb
$\nleq, \ngeq$

%Question 10
% environnement equation mathtools packets
\begin{equation*}
    e = mc^{2}
\end{equation*}

%Question 11
% la diffe c ets que equation c est numerotable
% donc on peut lui associer un label
\begin{equation*}
    e = mc^{2} 
\end{equation*}
\begin{equation*}
    cos(\frac{\pi}{2}) = 0
\end{equation*}

\begin{equation}
    \exp{0}=1
\end{equation}
\begin{equation}
    % Question 12 on voit la reference "1" de l equation
    e = mc^{2} \label{equation/ettiquette1}
\end{equation}
\begin{equation}
    cos(\frac{\pi}{2}) = 0
\end{equation}

\blindtext[2]
%Question 13 c est exactement la reference de l equationa vec la parenthese
% Qestion 14 sa a changer le num de referencement
et ensuite voici une reference à \eqref{equation/ettiquette1}

%Question 15
$\forall_{k} < n, \text{k est un diviseur de n!}$
\section{Question 16}
Trop long est inutile

%Question 17
\begin{equation*}
    \lim_{\theta \rightarrow (\frac{\pi}{2})^{-}} {tan(\theta) = \infty}
\end{equation*}

\end{document}