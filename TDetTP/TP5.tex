\documentclass{article}
\usepackage[utf8]{inputenc}
\usepackage[T1]{fontenc}
\usepackage{listings} % Pour le code (Question 4)
\usepackage{graphicx} % Pour les images (Question 6)
\usepackage{xcolor}   % Optionnel, mais rend le code plus joli

% Configuration globale ou par défaut pour listings (Question 4b, 4c, 4d)
\lstset{
    basicstyle=\small\ttfamily, % Fonte monospace et petite taille
    columns=fixed,              % Alignement strict des caractères
    tabsize=2,                  % Taille des tabulations
    frame=none,                 % Pas de cadre par défaut
    breaklines=true
}

\begin{document}

% Question 1 : Création de la section et de la liste
\section{Exercice « Mise en place »}

\begin{enumerate}
    % Question 2, 3 et 4 : Commandes Shell (arborescence) avec le paquet listings
    \item Commandes pour l'arborescence :
    % Attention : Remplace les commandes ci-dessous par celles que tu as vraiment utilisées
    % L'option gobble=4 permet d'ignorer les 4 espaces d'indentation du code LaTeX
    \begin{lstlisting}[language=bash, keywordstyle=\bfseries, gobble=4]
    mkdir repoC
    cd repoC
    touch correction.tex
    mkdir -p sous/dossier
    \end{lstlisting}

    % Question 5 : Commandes Shell (git version)
    \item Commandes pour la version git :
    \begin{lstlisting}[language=bash, keywordstyle=\bfseries, gobble=4]
    git init
    git add .
    git commit -m "Initialisation"
    \end{lstlisting}

    % Question 6 : Correction de la question 3 de l'exercice 1 (.gitignore et image)
    \item On vérifie l'état avec la commande \lstinline|git status|. % Question 6e
    
    Voici le contenu du fichier ignoré :
    % Question 6a, 6b, 6c : Import du fichier .gitignore
    %\lstinputlisting[frame=single, linewidth=.25\linewidth]{.gitignore}
    
    Et voici le résultat visuel :
    % Question 6d : Insertion de l'image (assure-toi d'avoir une image, ex: status.png)
    %\begin{center}
    %    \includegraphics[width=.5\linewidth]{capture_git_status.png}
    %\end{center}

    \item % Élément vide pour la question 4 (réservé pour la suite)
    \item % Élément vide pour la question 5 (réservé pour la suite)

\end{enumerate}

\end{document}