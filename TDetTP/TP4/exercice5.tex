\documentclass{article}

\usepackage[french]{babel}
\usepackage[T1]{fontenc}
\usepackage{lmodern}
\usepackage[utf8]{inputenc}
\usepackage{xspace}
\usepackage{mathtools}
\usepackage{amssymb}
\usepackage{microtype}
\usepackage{ragged2e}
\usepackage{hyperref}
\usepackage{cleveref}
\usepackage{amsthm}

\author{Mustafa}
\title{Graisse}
\date{\today}

%Question 6
\newcommand{\strong}[1]{{\bfseries #1}\xspace}

\begin{document}

\begin{titlepage}
    \maketitle
\end{titlepage}

%Question 5
\textbf{Mustafa}
\bfseries
Mustafa
\textmd{text en graisse moyenne}
\mdseries
text en graisse moyenne
\begin{flushleft}
    %Question 3
    Attention, \(2 + 2 = 2 \times 2\) mais \(3 + 3 \neq 3 \times 3\). \\
    %Question 4, il n y a que le text du mode math qui est en gras pas le reste
    \bfseries 
    Attention, \(2 + 2 = 2 \times 2\) mais \(3 + 3 \neq 3 \times 3\). \\
    %Question 5 TOUT le texte est en gras
    \mathversion{bold}

    Attention, \(2 + 2 = 2 \times 2\) mais \(3 + 3 \neq 3 \times 3\). \\

\strong{Bonjour monde}.
\end{flushleft}
\end{document}