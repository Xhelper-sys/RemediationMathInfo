\documentclass{article}

\usepackage[french]{babel}
\usepackage[T1]{fontenc}
\usepackage{lmodern}
\usepackage[utf8]{inputenc}
\usepackage{xspace}
\usepackage{mathtools}
\usepackage{amssymb}
\usepackage{microtype}
\usepackage{ragged2e}
% Question 5 : apporte des chagement au lien 
\usepackage[colorlinks=true, urlcolor=blue]{hyperref}

\usepackage{cleveref}
\usepackage{amsthm}
\usepackage{xcolor} % paquet qui sert a ecrire en couleur

%Question 6 
\usepackage{soul}
%Paquet qui ameliore et le surlignage
%et une amelioration de utf8
\author{Mustafa}
\title{Coloration}
\date{\today}

%Question 8
\newcommand{\surligne}[2][pink]{{\sethlcolor{#1}\hl{#2}}\xspace}
%Question 10
\newcommand{\bgfg}[3]{\sethlcolor{#1} \textcolor{#2}{\hl{#3}} }

\begin{document}

\begin{titlepage}
    \maketitle
\end{titlepage}

%Question 1
\textcolor{blue}{bleu}, 
\textcolor{red}{rouge},
\textcolor{green}{vert},
\textcolor{gray}{gris},
\textcolor{purple}{violet},
\textcolor{purple}{e}\textcolor{yellow}{t}\textcolor{pink}{c}

%Question3
\colorlet{firstnamecolor}{gray}
\colorlet{lastnamecolor}{cyan}

%Question 4 elle change la couleur de 
% la page courant
\pagecolor{white}

\newpage

\url{https://google.com} \\

\sethlcolor{green} % Question 7
\hl{vert} 
% Question 6 par defaut il surligne en 
% transparant

\surligne[pink]{hello}
\hl{vert?}

\bgfg{pink}{red}{blabla}


\end{document}