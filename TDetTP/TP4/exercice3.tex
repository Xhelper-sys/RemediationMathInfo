\documentclass{article}

\usepackage[french]{babel}
\usepackage[T1]{fontenc}
% Question 5
\usepackage{lmodern} % sert a avoir des caracetre svg
\usepackage[utf8]{inputenc}
\usepackage{xspace}
\usepackage{mathtools}
\usepackage{amssymb}
\usepackage{microtype}
\usepackage{ragged2e}
\usepackage{hyperref}
\usepackage{cleveref}
\usepackage{amsthm}

%% Question 4
%% Question 3
%\usepackage{times} %permet d'introduire
% d autre polices 



\author{Mustafa}
\title{Définition de commande}
\date{\today}

\begin{document}
\begin{titlepage}
    \maketitle
\end{titlepage}

%Question 1
\begin{flushleft}
\textrm{Je suis en police par défaut.} 
\textsf{Je suis en police avec empattement.}
\texttt{Je suis en police à chasse fixe.}
Je suis en police roman, explicitement    
\end{flushleft}

% Question 2
\begin{enumerate}
    \item  Je ne change la police par défaut...
    \item ...qu’à partir de \sffamily maintenant, pour une police sans empattement...
    \item ...qui continue un petit moment...
    \item ...avant d’être \rmfamily finalement changée...
    \item ... pour la police par défaut...
    \item ... puis pour \ttfamily la police à chasse fixe...
    \item ... qui continue...
    \item ... encore et encore, \rmfamily mais pas jusqu’au bout.
\end{enumerate}
Après la liste, je retrouve la police par défaut.
\end{document}