%Exercice 5
\documentclass{article}

\usepackage[french]{babel}
\usepackage[T1]{fontenc}
\usepackage[utf8]{inputenc}
\usepackage{mathtools}
\usepackage{amssymb}
\usepackage{upquote} % Les symbole de guillemets plsu precis

\title{Quantificateur en tableau}
\author{Mustafa Kaya}
\date{\today}

\begin{document}
\begin{titlepage}
    \maketitle
\end{titlepage}

% 1 Les sigles of et fg signifie les les guillemets francais
% ouvrant et fermant et c'est commande ne sont pas définis sans
% l impport de french babel
\begin{center}
TABLE 1 - Tableau de guillemets
\begin{tabular}{c c c c c}
    \textbf{nom} & \textbf{nom anglais} & \textbf{utf8} & \textbf{code \LaTeX} & \textbf{rendu} \\ \hline
    simples guillemets & simple quotes & `hey' & \verb|`hey`| & `hey' \\
    doubles guillemts & double quotes & ``hey'' & \verb|``hey''| & ```hey'' \\
    guillemets francais & french quotes & \og hey\fg & \verb|\og hey\fg| & \og hey\fg

\end{tabular}
\end{center}
\end{document}