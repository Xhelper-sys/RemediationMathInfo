%exercice 3

\documentclass{article}

\usepackage[utf8]{inputenc}
\usepackage[T1]{fontenc}
\usepackage[french]{babel}
\usepackage{mathtools}
\usepackage{amssymb}

%Question 3
\usepackage{amsthm} % paquet qui fournit des commandes 
%qui inclues des environnements theorem/ proof = preuve ....

\title{Mathématique discret}
\author{Mustafa Kaya}
\date{\today}

\newtheorem{theorem}{Théorème}

\begin{document}
\begin{titlepage}
    \maketitle
\end{titlepage}
%Question 2
$\forall_{n} \in \mathbb{N}$

\begin{theorem}
    Pour tout entier n, la somme des n premiers entier 
    naturels est égal à la moitié du produit de n par $(n - 1)$.

\end{theorem}









% Question 5
\begin{proof}
    Soit $n \in \mathbb{N}$.
    Notons $S_n$ la somme des $n$ premiers entiers naturels.
    Alors :
    \begin{align*}
        S_n &= \sum_{i = 0}^{n - 1} i
        = 0 + 1 + \cdots + (n - 2) + (n - 1)
        & \text{par définition} \\
        &= \sum_{i = 0}^{n - 1} (n - 1 - i)
        = (n - 1) + (n - 2) + \cdots + 1 + 0
        & \text{par commutativité} \\
    \end{align*}

    En sommant les deux formes de la somme $S_n$ données ci-dessus,
    nous pouvons obtenir :
    \begin{align*}
        2 \times S_n
        &= \left( \sum_{i = 0}^{n - 1} i \right)
        + \left( \sum_{i = 0}^{n - 1} (n - 1 - i) \right) \\
        &= \sum_{i = 0}^{n - 1} (i + (n - 1 - i))
        & \text{par associativité et commutativité} \\
        &= n(n-1) & \text{On divise ensuite par 2 des deux coté}\\
        &= \frac{2 \times S_{n}}{{2}} = \frac{n(n-1)}{2}
    \end{align*}

    \emph{La somme de 0 jusqu'aà n-1 entier naturel est donc : $\frac{n(n-1)}{2}$}
\end{proof}

%Question 11 et 12 flemme 
\end{document}